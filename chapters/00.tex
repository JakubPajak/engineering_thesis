\subsubsection*{Tytuł pracy} 
\Title

\subsubsection*{Streszczenie}  
Celem projektu jest wykonanie autonomicznej platformy jezdnej, mającej na celu
bezobsługowe sortowanie klocków spadających w określone miejsce z taśmociągu lub
innego podajnika. Robot korzystając z możliwości wizji komputerowej będzie
rozpoznawać kolor aktualnego klocka, a następnie na podstawie określonego koloru
będzie wybierać trasę do odpowiedniego pojemnika. Domyślnie rozpoznawane będą trzy
kolory (czerwony, zielony, niebieski) oraz trzy pojemniki, odpowiednie dla każdego koloru.
Wymagania: znajomość programowania układów SBC (Raspberry Pi), Python, OpenCV


\subsubsection*{Słowa kluczowe} 
Robot mobilny, analiza wizyjna, Programowanie mikrokontrolerów

\subsubsection*{Thesis title} 
\begin{otherlanguage}{british}
\TitleAlt
\end{otherlanguage}

\subsubsection*{Abstract} 
\begin{otherlanguage}{british}
    The goal of this project is to create an autonomous mobile platform designed for
    unattended sorting of blocks falling into a designated area from a conveyor belt or
    another feeder. The robot will use computer vision to recognize the color of the
    current block and, based on the identified color, will choose the path to the
    appropriate container. By default, three colors (red, green, blue) will be recognized,
    and there will be three containers, each corresponding to one of the colors.
    Requirements: knowledge of SBC programming (Raspberry Pi), Python, OpenCV
\end{otherlanguage}
\subsubsection*{Key words}  
\begin{otherlanguage}{british}
Mobile robot, visual analysis, microcontroller programming
\end{otherlanguage}

