\chapter{Analiza tematu}
\label{ch:analiza}

W rozdziale tym zostanie przeanalizowany temat projektu, z uwzględnieniem aktualnego stanu wiedzy oraz znanych rozwiązań dotyczących automatycznych systemów sortowania za pomocą wizji komputerowej.

\section{Sformułowanie problemu}
\label{sec:sformulowanie_problemu}

Głównym problemem rozwiązywanym w tym projekcie jest automatyzacja procesu sortowania obiektów według ich cech wizualnych, w tym przypadku koloru. Tradycyjne metody sortowania wymagają interwencji człowieka lub stosowania mechanicznych sorterów, które często są mniej elastyczne i mniej precyzyjne w identyfikacji złożonych cech. Celem projektu jest wykorzystanie technologii wizji komputerowej, aby umożliwić robotowi autonomiczne rozpoznawanie i sortowanie klocków według koloru.

\section{Osadzenie tematu w kontekście aktualnego stanu wiedzy (\textit{state of the art})}
\label{sec:state_of_the_art}

Temat projektu osadza się w dziedzinach robotyki mobilnej oraz wizji komputerowej, które rozwijają się bardzo dynamicznie w ostatnich latach. Rozwiązania oparte na wizji komputerowej, zwłaszcza w połączeniu z platformami SBC (Single-Board Computer), takimi jak Raspberry Pi, stają się coraz bardziej popularne w automatyzacji procesów przemysłowych i logistyce. Wykorzystanie takich narzędzi jak OpenCV w projektach robotycznych pozwala na zwiększenie elastyczności i precyzji systemów sortujących.

\section{Studia literaturowe}
\label{sec:studia_literaturowe}

Studia literaturowe będą obejmowały analizę dostępnych rozwiązań naukowych i technicznych dotyczących autonomicznych systemów sortowania oraz technologii wizji komputerowej. Przegląd literatury pozwoli zidentyfikować kluczowe algorytmy wykorzystywane do rozpoznawania kolorów oraz technologie stosowane w systemach robotycznych tego typu. Ważne będzie także porównanie różnych podejść do implementacji i optymalizacji systemów autonomicznych sorterów.

\cite{bib:manualarduino,bib:manualLibcamera2,bib:artykul,bib:artykul1,bib:artykul2,bib:ksiazka,bib:konferencja,bib:internet}
