\chapter{Wstęp}
\label{ch:wstep}

Niniejsza praca inżynierska stanowi propozycję rozwiązania problemu automatycznego sortowania klocków na podstawie analizy wizyjnej, a dokładnie rozpoznania koloru. Temat pracy jest niezwykle aktualny, ze względu na bardzo dynamicznie rozwijający się przemysł 4.0 oraz powrzechną robotyzację i automatyzację procesów. Projekt ten wpisuje się w trend stwarzając możliwość wykorzystania go w szeroko pojętym przemyśle oraz powierzchniach magazynowych. 

W tym rozdziale zostanie omówiony problem, którego rozwiązaniem jest robot mobilny, osadzenie problemu w dziedzinie, cel pracy oraz krótki opis poszczególnych rozdziałów. 

\section{Wprowadzenie w problem}
\label{sec:wprowadzenie}

Proces sortowania przedmiotów jest zadaniem, do którego wykonania konieczni byli ludzie, ze względu na umiejętność klasyfikowania obiektów na podstawie ich wizualnych aspektów. Wraz z rozwojem wizji komputerowej, możliwe stało się, aby oprogramowanie dokonywało podobnej klasyfikacji. Obecnie zaawansowane algorytmy uczenia maszynowego nierzadko są bardziej skuteczne od ludzi. 

Jednak, żeby przeprowadzić proces sortowania w dynamicznym środowisku, obecność medium transportowego jest wymagana. W tym miejscu na przeciw wychodzi robotyka. Połączenie platformy mobilnej wraz z wizją komputerową pozwala na uzyskanie bardzo skutecznej formy sortowania obiektów. 

Zastosowanie mobilnej platformy wyposażonej w kamerę jest szczególnie przydatne w dynamicznym środowisku przemysłowym lub magazynowym, gdzie istotna jest elastyczność i możliwość dostosowania trasy robota do aktualnych warunków. 

Sam fakt automatyzacji procesu sortowania niesie ze sobą dużo krzyści takich jak: zmniejszenie kosztów operacyjnych, zwiększenie efektywności oraz redukcję błędów ludzkich w procesie sortowania.  

\section{Osadzenie problemu w dziedzinie}
\label{sec:osadzenie}

Projekt można osadzić w dziedzinie robotyki mobilnej wspartej wizją komputerową. Roboty mobilne, zdolne do samodzielnej lub częściowo samodzielnej nawigacji oraz wykonywania skomplikowanych zadań, znajdują coraz szersze zastosowanie w przemyśle lub logistyce. W kontekście niniejszego projektu, kluczową rolę odgrywa zastosowanie odometrii oraz systemów napędowych, które pozwalają na precyzyjną kontrolę ruchu robota.

Wizja komputerowa, z kolei, umożliwia rozpoznawanie obiektów na podstawie ich cech wizualnych. Technologia ta, oparta na bibliotekach takich jak OpenCV, stała się dostępna nawet dla małych systemów wbudowanych, na przykład Raspberry Pi. W niniejszym projekcie robot wykorzystuje kamerę do rejestrowania obrazów klocków, które następnie są analizowane w czasie rzeczywistym w celu określenia koloru, a na tej podstawie podejmowana jest decyzja gdzie bieżący klocek powinien zostać przetransportowany.

\section{Cel pracy}
\label{sec:cel}

Celem projektu jest zaprojektowanie oraz zbudowanie platformy jezdnej pracującej w trybie automatycznym. Ponadto celem jest implementacja systemu kontroli robota, umożliwiającego samodzielne podejmowanie decyzji o wyborze miejsca docelowego dla danego obiektu.

Istotnym założeniem poprawności działania platformy jest dokładność i powtarzalność działania, wpływająca na możliwość automatycznej pracy bez ingerencji człowieka.

Dodatkowo system powinien być elastyczny, umożliwiający łatwe rozszerzenie o dodatkowe funkcje, takie jak rozpoznawanie większej liczby kolorów lub innych cech klocków.


\section{Zakres pracy}
\label{sec:zakres}

Projekt obejmuje szeroki zakres działań, zarówno w zakresie sprzętowym, jak i programistycznym. Poniżej przedstawiono główne etapy realizacji:

\begin{itemize}
    \item Zbudowanie fizycznej platformy jezdnej, w tym konstrukcji mechanicznej oraz systemu napędowego.
    \item Implementacja systemu rozpoznawania kolorów za pomocą kamery oraz algorytmów wizji komputerowej z użyciem Raspberry Pi i biblioteki OpenCV.
    \item Opracowanie algorytmu podejmowania decyzji na podstawie rozpoznanego koloru i przekierowania klocka do odpowiedniego pojemnika.
    \item Testowanie i optymalizacja działania systemu, aby zapewnić jego poprawne funkcjonowanie w rzeczywistych warunkach.
\end{itemize}

\section{Zwięzła charakterystyka rozdziałów}
\label{sec:charakterystyka}

Struktura pracy została podzielona na następujące rozdziały:

\begin{itemize}
    \item \textbf{Rozdział 1: Wstęp} – Przedstawienie problemu, celów oraz zakresu projektu.
    \item \textbf{Rozdział 2: Analiza tematu} – Przegląd dostępnych rozwiązań, technik oraz technologii stosowanych w podobnych systemach, a także omówienie aktualnego stanu wiedzy w tej dziedzinie.
    \item \textbf{Rozdział 3: Założenia wstępne} – Opis wymagań projektowych, przedstawienie diagramów funkcjonalnych, opis wykorzystanego sprzętu elekronicznego oraz metodyka pracy. 
    \item \textbf{Rozdział 4: Podstawy teoretyczne} – Przedstawienie teoretycznych podstaw działania wykorzystanych techonologii. 
    \item \textbf{Rozdział 5: Wymagania i specyfikacja użytkowa} – Opis funkcjonalności systemu z perspektywy użytkownika,w tym sposobu instalacji oraz wykorzystania. 
    \item \textbf{Rozdział 6: Specyfikacja techniczna} – Szczegółowa analiza implementacji systemu, w tym struktury oprogramowania i integracji z komponentami sprzętowymi.
    \item \textbf{Rozdział 7: Weryfikacja i walidacja} – Opis przeprowadzonych testów, ocena efektywności systemu oraz zgodności z założeniami projektowymi.
    \item \textbf{Rozdział 8: Podsumowanie i wnioski} – Zakończenie projektu, omówienie osiągniętych celów, a także wskazanie możliwych kierunków rozwoju systemu.
\end{itemize}

% \section{Wkład autora}
% \label{sec:wklad}

% Projekt został w pełni zrealizowany przez autora. Autor odpowiada za wszystkie etapy pracy, począwszy od projektowania i budowy platformy, poprzez implementację algorytmów rozpoznawania kolorów i sterowania ruchem robota, aż po testowanie i optymalizację systemu. Ponadto, autor zaprojektował system kontrolny, który pozwala na monitorowanie pracy robota w czasie rzeczywistym oraz umożliwia wprowadzenie ewentualnych korekt.
