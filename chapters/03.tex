\chapter{Wymagania i narzędzia}
\label{ch:wymagania-i-narzedzia}

W rozdziale tym przedstawiono szczegółowy opis wymagań funkcjonalnych i niefunkcjonalnych, które określają, jakie zadania system musi spełniać oraz jakie kryteria jakościowe są wymagane do zapewnienia prawidłowego działania platformy mobilnej do sortowania klocków według koloru. Omówiono również narzędzia i metody, które wspierają realizację projektu, w tym wykorzystywane technologie, frameworki oraz metodologię pracy nad projektem.

\section{Wymagania funkcjonalne i niefunkcjonalne}
Wymagania funkcjonalne odnoszą się do kluczowych funkcji systemu, które muszą zostać spełnione, aby system mógł realizować swoje podstawowe zadania. W przypadku autonomicznej platformy mobilnej do sortowania klocków funkcje te obejmują między innymi:
\begin{itemize}
    \item Rozpoznawanie kolorów klocków – system musi identyfikować kolory klocków przy pomocy kamery i odpowiednich algorytmów przetwarzania obrazu (czerwony, zielony, niebieski).
    \item Przekierowywanie klocków – po rozpoznaniu koloru, system kieruje klocek do odpowiedniego pojemnika.
    \item Samodzielne poruszanie się – robot powinien nawigować w wyznaczonej przestrzeni zgodnie z zaprogramowanymi trasami.
\end{itemize}

Wymagania niefunkcjonalne obejmują kryteria jakościowe, które determinują, jak system ma działać. Są to:

\begin{itemize}
    \item Niezawodność – system powinien pracować nieprzerwanie przez określony czas bez błędów.
    \item Szybkość przetwarzania obrazu – analiza obrazu i detekcja koloru muszą odbywać się na bieżąco, co jest szczególnie istotne w przypadku szybkich linii produkcyjnych.
\end{itemize}

\section{Przypadki użycia i diagramy UML}


\section{Opis narzędzi i metod}
W projekcie wykorzystano różnorodne narzędzia wspomagające rozwój systemu, między innymi:

\begin{itemize}
    \item Raspberry Pi – służy jako główny komputer zarządzający systemem, wyposażony w odpowiednie moduły do obsługi kamery oraz komunikacji.
    \item Arduino UNO R3 - mikrokontroler pełniący rolę kontrolera napędów oraz enkoderów. 
    \item OpenCV – biblioteka do przetwarzania obrazu. 
    \item Python - główny język programowania obsługujący system kontroli robota oraz analizę wyzyjną. 
    \item Libcamera2 – biblioteka wspomagająca obsługę kamer podłączonych przez interfejs CSI, co umożliwia komunikację między Raspberry Pi a kamerą.
    \item Raspbian - system operacyjny układu SBC Raspberry Pi. Został on wykorzystany ze względu na kompatybilność z komputerem. 
    \item Visual Studio Code - środowisko rozbudowanego edytora tekstu, pełniącego rolę programu do pisania aplikacji oraz zdalnego łączenia się z komputerem poprzez protokuł SSH. 
    \item Arduino IDE - zaintegrowane środowisko programistyczne dedykowane do programowania mikrokontrolerów rodziny ATMega. 
\end{itemize}


\section{Metodyka pracy nad projektowaniem i implementacją}
Realizacja projektu przebiegała zgodnie z metodyką iteracyjno-przyrostową, co pozwalało na stopniowe dodawanie funkcjonalności i testowanie każdego etapu w rzeczywistych warunkach pracy systemu. Na każdym etapie projektowania, implementacji i testowania zbierano informacje o działaniu systemu i, jeśli zachodziła potrzeba, dostosowywano parametry systemu lub optymalizowano kod.